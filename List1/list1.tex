\documentclass{beamer}
\usetheme{Boadilla}
\usepackage{algorithm,algorithmic}
\usepackage{hyperref}
\title{CSCI 5610 Solutions to Exercises(List 1)}
\author{Anonymous}
\begin{document} 
\begin{frame}
    \titlepage
\end{frame}
\begin{frame}
	\frametitle{Lecture 2: Problem 1}
	1. For the $(-\infty,y)$ case: \\
	\begin{enumerate}
		\item Initially, $ \Sigma = \Phi \ $, and set $ u $ to the root of $ T $ .
		\item If the key of $ u $ equals $ x $, then add $slab(lc(u)))$ to $ \Sigma $, and stop. (lc: left child,rc: right child)
		\item If the key of $ u  <  x $, then add $slab(lc(u)))$ to $ \Sigma $, and let $u = rc(u)$.
		\item If the key of $ u  >  x $, let $u = lc(u)$, repeat from 2.
	\end{enumerate}
	2. For the $ [x, y) $ case, we calculate canonical slabs of $ [x, \infty] $ and $ (-\infty,y).$ Then we remove the two unbounded intervals containing $ \infty  $ and $ -\infty $.
\end{frame}



\begin{frame}
	\frametitle{Lecture 2: Problem 2}
	Construct a BST on the age of people with each node u containing the following additional information:(1) the range of ages covered in the subtree rooted on $ u $, namely l, r; (2) the maximum salary in the subtree rooted on u.\\
	For a query $ [x,y] $, we will handle it in a recursive way:
	\begin{algorithm}[H]
		\begin{algorithmic}[1]
			\IF{$y < u.l$ \OR $x > u.r$} \STATE {return $-\infty$} \ENDIF
			\IF{$ x \leq u.l $ \AND $ y \geq u.r $} \STATE {return u.maxSalary} \ENDIF
			\IF{($ x \geq u.l $ \AND $ x \leq u.r $) \OR ($ y \geq u.l $ \AND $ y \leq u.r $)} \STATE {return max(queryMax(lc(u), x, y),queryMax(rc(u), x, y) ))} \ENDIF
		\end{algorithmic}
		\caption{queryMax(u,x,y)}
		
	\end{algorithm}

\end{frame}


\begin{frame}
	\frametitle{Lecture 2: Problem 2 Con't}
	As for the query time of O(log n), we can see that no more than 4 nodes will be visited at each level and there are O(log n) levels in total. (\href{https://www.quora.com/How-do-we-prove-that-the-time-complexity-of-the-update-and-query-operations-in-a-segment-tree-are-Theta-lg-n}{Click here for reference} )
	
\end{frame}

\begin{frame}
	\frametitle{Lecture 2: Problem 3}
	In each node u, we store the size of the subtrees rooted on lc(u) and rc(u) as l\_cnt and r\_cnt correspondingly. In order to find the k-th largest element in S, we define the following function:
		\begin{algorithm}[H]
		\begin{algorithmic}[1]
			\IF{$u.l\_cnt \geq k$ } \STATE {return find\_kth\_largest(lc(u),k)} \ENDIF
			\IF{$u.l\_cnt + 1 == k$ } \STATE {return u.key} \ENDIF
			\IF{$u.l\_cnt + 1 \leq k$ } \STATE {return find\_kth\_largest(rc(u),k-(u.l\_cnt + 1))} \ENDIF
		\end{algorithmic}
		\caption{find\_kth\_largest(u,k)}
		
	\end{algorithm}
	
\end{frame}

\begin{frame}
	\frametitle{Lecture 2: Problem 4}
	To get a 2-3 tree on the set S\textbackslash [x,y], we can use the existing split and join functions. 
	\begin{enumerate}
		\item Use x to split S into $ S_1 $ and $ S_2$. All the values in $ S_1 $ is smaller than x. We get a 2-3 tree on $ S_1 $ and a 2-3 tree on $ S_2 $. 
		\item Use y to split $ S_2 $ into $ S_{21} $ and $ S_{22} $ in the same way. 
		\item Delete y in the 2-3 tree on $ S_{22} $ if it exists.
		\item Join $ S_1 $ and $ S_{22} $ to get a 2-3 tree on set S\textbackslash [x,y]
	\end{enumerate} 
	
	
\end{frame}

\begin{frame}
	\frametitle{Lecture 2: Problem 5*(optional)}
	
	
\end{frame}

\end{document}
